% Created 2023-04-08 六 14:40
% Intended LaTeX compiler: xelatex
\documentclass[11pt]{article}
\usepackage{graphicx}
\usepackage{longtable}
\usepackage{wrapfig}
\usepackage{rotating}
\usepackage[normalem]{ulem}
\usepackage{amsmath}
\usepackage{amssymb}
\usepackage{capt-of}
\usepackage{hyperref}
\usepackage[UTF8]{ctex}
\usepackage{geometry}
\geometry{a4paper, scale=0.8}
\linespread{1}
\author{不要在意我的头像QwQ \href{https://youlanjie.github.io/}{Chglish}}
\date{\today}
\title{八下道法背诵材料}
\hypersetup{
 pdfauthor={不要在意我的头像QwQ \href{https://youlanjie.github.io/}{Chglish}},
 pdftitle={八下道法背诵材料},
 pdfkeywords={},
 pdfsubject={},
 pdfcreator={Emacs 28.2 (Org mode 9.5.5)}, 
 pdflang={English}}
\begin{document}

\maketitle
\tableofcontents

\begin{verse}
注意!本材料由我纯手工打入,难免会有错字。请谅解。\\
内容取自 \emph{《广东中考 高分突破》道德与法治 八年级下册 知识宝典}\\
\end{verse}

\section{维护宪法权威}
\label{sec:org49e4a5b}
\subsection{党的主张和人民意志的统一}
\label{sec:orgb05f81a}
\subsubsection{新中国成立后,人民是如何确认奋斗成果的?}
\label{sec:org3f49698}
新中国成立后,中国共产党领导人民制定宪法,以法律的形式确认了中国各族人民奋斗的成
果,确立了在历史和人民的选择中形成的中国共产党的领导地位。我国的宪法是党的主张和
人民意志的统一。
\subsubsection{中国共产党的性质、根本宗旨、目标和地位分别是什么?}
\label{sec:org750cda5}
\begin{enumerate}
\item 性质:中国共产党是中国工人阶级的先锋队,同时是中国人民和中华民族的先锋队。
\item 根本宗旨:全心全意为人民服务
\item 地位:中国共产党是中国特色社会主义事业的领导核心。党政军民学,东南西北中,党
是领导一切的。
\end{enumerate}
\subsubsection{坚持中国共产党领导的重要性和做法分别是什么?}
\label{sec:orga314aa1}
\begin{enumerate}
\item 重要性:中国共产党领导是中国特色社会主义最本质的特征,是中国特色社会主义制度
的最大优势。是党和国家的根本所在,命脉所在,是全国各族人民的利益所系、命运所
系。党是最高政治领导力量。
\item 做法:必须坚决维护党中央权威和集中统一领导,健全总览全局、协调各方的党的领导
制度体系,把党的领导落实到党和国家事业各领域各方面各环节。
\end{enumerate}
\subsubsection{中国共产党如何坚持依宪执政?}
\label{sec:org53948b9}
中国共产党领导人民制定宪法和法律,领导人民实施宪法和法律,做到党领导立法、保证执
法、支持司法、带头守法。中国共产党要履行好执政兴国的重大职责,必须在宪法和法律范
围内活动,依据宪法和法律治国理政。
\subsubsection{我国宪法的基本原则是什么?}
\label{sec:orga5b38a0}
我国是人民民主专政做的社会主义国家。国家的一切权力属于人民,这是我国宪法的基本原
则。这一原则归根结底是要保证人民当家做主的权利。
\subsubsection{宪法如何确保国家一切权力属于人民?}
\label{sec:org9d1e1d2}
\begin{enumerate}
\item 宪法确认我国的 \uline{国家性质} ,明确人民当家做主的地位。
\item 宪法规定的 \uline{社会主义经济制度} ,奠定了国家权力属于人民的经济基础。
\item 宪法规定的 \uline{社会主义政治制度} ,明确了人民行使国家权力的基本途径和形式。
\item 宪法规定 \uline{广泛的公民基本权利} ,并规定实现公民基本权利的保障措施。
\item 宪法规定 \uline{国家武装力量属于人民} ,并规定其任务是巩固国防,抵抗侵略,保卫祖国,
保卫人民的和平劳动,参加国家建设事业,努力为人民服务。\footnote{宪法第二十九条}
\end{enumerate}
\subsubsection{最大的人权、我国人权的主体、保护对象及内容分别是什么?}
\label{sec:org396d3ab}
\begin{enumerate}
\item 最大的人权:人民幸福生活是最大的人权
\item 主体、保护对象:在我国,人权的主体非常广泛,既包括我国公民,也包括外国人等。
人权保护的对象不仅包括个人,也包括群体。
\item 内容:宪法保护的人权内容也很广泛,既包括 \uline{平等权} 和 \uline{人身权利} 、 \uline{政治权利}
,也包括 \uline{财产权} 、 \uline{劳动权} 、 \uline{受教育权} 等 \uline{经济社会} 、 \uline{文化方面} 的权
利。
\end{enumerate}
\subsubsection{国家如何做到尊重和保障人权?}
\label{sec:orgfafea0d}
\begin{enumerate}
\item 各级国家机关树立尊重和保障人权的理念,加强人权法治保障,保证人民依法享有广泛
权利和自由。
\item 尊重和保障人权是立法活动的基本要求。我国宪法规定了公民享有的广泛的基本权利,
法律进一步明确了公民享有的各项具体权利,规定了侵害权利的法律责任。
\item 行政机关在执法过程中应当树立尊重和保障人权的意识,做到严格规范公正文明执法,
坚持依宪施政、依法行政、简政放权。
\item 监察机关依照法律规定独立行使监察权,加强对所有行使公权力的公职人员的监督,保
护公民的各项合法权益。
\item 审判机关、检察机关要依照宪法和法律的规定分别独立行使审判权、检察权,保护公民
的各项合法权益。
\item 国家加强法治宣传教育,弘扬社会注意法治精神,建设社会主义法治文化,增强全民法
治观念,形成全民守法的氛围和习惯,努力将人权理想变成现实。
\end{enumerate}
\subsection{治国安邦的总章程}
\label{sec:orgbe456bc}
\subsubsection{宪法如何设置国家机构?}
\label{sec:org3dea87a}
\begin{enumerate}
\item 设置原则:宪法明确国家机构实行民主集中制的原则\footnote{宪法第三条}
\item 设置方式:宪法明确各级国家机关的产生。人民代表大会是人民行使国家权力的机关。
全国人民代表大会和地方各级人民代表大会都由民主选举产生,对人民负责,受人民监
督。国家行政机关、监察机关、审判机关、检察机关都由人民代表大会产生,对它负责,
受它监督。
\item 规范权力运行:宪法授予国家机构特定职权,明确国家机构的组成、任期、工作方式等
内容,使得国家权力的运行稳定有序。国家机构依据宪法行使权力,以实现和维护人民
的根本利益。
\end{enumerate}
\subsubsection{国家机构贯彻民主集中制原则主要体现在那些方面?}
\label{sec:org7d3ae2f}
\begin{enumerate}
\item 在国家机构与人民的关系方面,国家权力来自人民,由人民选举产生国家权力机关,国
家权力机关在国家机构中居于主导地位。
\item 在中央与地方国家机构的关系方面,中央和地方的国家机构职权的划分,遵循在中央的
统一领导下,充分发挥地方的主动性、积极性的原则\footnote{宪法第三条}
\item 在国家机关内部作出决策、决定时,实行民主集中制
\end{enumerate}
\subsubsection{为什么要规范权力运行?}
\label{sec:orgb511c63}
\begin{enumerate}
\item 权力是把双刃剑,运用得好,可以造福于民;如果被滥用,则会滋生腐败,贻害无穷。
\item 规范国家权力运行以保障公民权利,这是宪法的核心价值追求。
\item 只有依法规范权力运行,才能保证人民赋予的权力始终用来为人民谋利益。
\end{enumerate}
\subsubsection{宪法如何规范权力的运行?}
\label{sec:org2f8887c}
\begin{enumerate}
\item 加强对权力运行的制约和监督,让人民监督权力,让权力在阳光下运行,把权力关进制
度的笼子。
\item 国家权力必须在宪法和法律限定的范围内行使,不能超越权限、滥用职权,否则要承担
法律责任。 \emph{(法无授权不可为)}
\item 国家级机关及其工作人员必须依法行使权力、履行职责,不能懈怠、推诿。 \emph{(法定职责
必须为)}
\item 国家权力必须严格按照法定的途径和方式行使。凡不按法定程序行使权力的行为都是违
法行为。 \emph{(按法定程序行使)}
\end{enumerate}
\section{保障宪法实施}
\label{sec:org0627dd7}
\subsection{坚持依宪治国}
\label{sec:orge092b5c}
\subsubsection{我国现行宪法的构成是怎样的?}
\label{sec:org873e096}
我国现行宪法除序言外,设有第一章总纲,第二章公民的基本权利和义务,第三章国家机构,
第四章国旗、国歌、国徽、首都,共四章一百四十三条。
\subsubsection{为什么要维护宪法权威?}
\label{sec:orgecbf9f9}
\begin{enumerate}
\item 我国宪法是党和人民意志的集中体现,是国家的根本法
\item 宪法的权威关系国家的命运、社会的安定和人民的根本利益。如果宪法没有权威,法治
的权威就树立不起来;如果宪法受到漠视,人民的权利和自由就无法保证。
\end{enumerate}
\subsubsection{如何维护宪法权威?}
\label{sec:orgdab2a68}
\begin{enumerate}
\item 坚持依法治国首先要坚持依宪治国,坚持依法执政首先要坚持依宪执政。\footnote{\emph{依法治国才是基本原则!}}
\item 任何公民、社会组织和国家机关都必须以宪法和法律为行为准则。一切组织和个人都必
须维护宪法权威,保障宪法实施,依照宪法和法律行使权利或权力,履行义务或职责,
都不得有超越宪法和法律的特权,一切违反宪法和法律的行为都必须予以追究。
\end{enumerate}
\subsubsection{为什么说宪法具有最高的法律效力?(宪法与其他法律的关系是怎样的?)}
\label{sec:orgfd405f3}
\begin{center}
\begin{tabular}{lll}
\hline
\hline
比较 & 宪法 & 其他法律\\
\hline
\hline
规定内容 & 规定国家生活中带有全局性、 & 规定国家生活中的一般性问题,是对国家\\
 & 根本性的问题 & 生活和社会生活中某一方面的规定\\
\hline
法律效力 & 具有最高的法律效力,是其他 & 是根据宪法指定的,不得与宪法的原则和\\
 & 法律的立法基础和立法依据 & 精神相违背,否则会因为违宪而无效\\
\hline
制定和修改程序 & 比其他法律更加严格: & \\
 & 1) 制定:遵循特定的制宪程序 & 依据一般程序,由全国人民代表大会以\\
 & 2) 修改:由全国人大常委会或 & 全体代表的过半数通过或者由全国人大\\
 & 五分之一以上的全国人大代表 & 常委会全体组成员的过半数通过\\
 & 提议,并由全国人大以全体代 & \\
 & 表的三分之二以上的多数通过 & \\
\hline
\hline
联系 & 1) 宪法是国家法制统一的基础, & 是中国特色社会主义法律体系的核心\\
 & 2) 宪法的规定具有原则性的特点 & 各种法律制度是对宪法规定的具体落实\\
 & 3) 宪法是对公民基本权利的根本 & 确认和保障,其他法律也对公民基本权利\\
 & 的实现具有不可替代的作用 & \\
\hline
\hline
\end{tabular}
\end{center}
\subsubsection{为什么说宪法是国家法治统一的基础,是中国特色社会主义法律体系的核心?}
\label{sec:org4a1c548}
\begin{enumerate}
\item 宪法的规定具有原则性的特点,各种法律制度是对宪法规定的具体落实。
\item 宪法是对公民基本权利的根本确认和保障,其他法律也对公民基本权利的实现具有不可
替代的作用。
\end{enumerate}
\subsubsection{为什么要坚持依宪治国?(宪法的地位/重要性)}
\label{sec:org6fd92ce}
\begin{enumerate}
\item 我国宪法是党的主张和人民意志的统一,是治国安邦的总章程。
\item 宪法是党和人民意志的集中体现,是国家的根本法。
\item 宪法具有至高无上的权威。
\item 宪法在国家法律体系中具有最高的法律地位、法律权威和法律效力。
\item 宪法是国家法治统一的基础,是中国特色社会主义法律体系的核心。
\end{enumerate}
\subsection{加强宪法监督}
\label{sec:orgff32e0a}
\subsubsection{宪法监督的主体和主要内容分别是什么?}
\label{sec:org0d54179}
\begin{enumerate}
\item 主体:全国人大及其常委会、地方各级人大
\begin{enumerate}
\item 全国人大及其常委会行使监督宪法实施的职权,全国人大常委会有权解释宪法和法律
\item 地方各级人大在本行政区域内负有保证宪法和法律实施的职责
\end{enumerate}
\item 主要内容:
\begin{enumerate}
\item 合宪性审查和监督,使法律、法规与宪法不抵触
\item 审查国家机关及其工作人员的违宪行为,追究其违宪责任,维护宪法权威
\end{enumerate}
\end{enumerate}
\subsubsection{为什么要加强宪法监督?}
\label{sec:org9332999}
\begin{enumerate}
\item 权力行使需要接受监督。监督是权力行使的根本保证,不受监督的权力将导致腐败。
\item 在监督公权力行使的制度体系中,宪法监督制度具有基础性意义
\item 加强宪法监督是保障宪法实施的重要举措
\item 健全宪法实施和监督制度,不断加强宪法监督工作,是全面依法治国的需要
\end{enumerate}
\subsubsection{如何加强宪法监督?}
\label{sec:org5e3c5c9}
\begin{enumerate}
\item 需要加强宪法实施和监督,健全保证宪法全民实施的制度体系
\begin{enumerate}
\item 完善全国人大及其常委会宪法监督制度,健全监督机制和程序,使其更好地负担起宪
法监督职责
\item 健全宪法解释程序机制,推进合宪性审查工作,加强备案审查制度和能力建设,加强
对宪法实施情况的监督检查,维护宪法权威
\item 对各种违宪行为都必须予以追究和纠正
\end{enumerate}
\item 需要人们增强宪法意识。如设立国家宪法日,建立宪法宣誓制度等。
\end{enumerate}
\subsubsection{我们应如何增强宪法意识?}
\label{sec:org642b5ce}
\begin{enumerate}
\item 学习宪法
\begin{enumerate}
\item 了解宪法产生和发展的历程,理解宪法的主要内容,领会宪法的原则和精神。
\item 积极参与宪法宣传活动,为增强全社会的宪法意识贡献自己的力量
\end{enumerate}
\item 认同宪法。理解并认同宪法的价值,增强对宪法的信服和尊崇,自觉接受宪法的指引与
要求,让宪法真正铭刻于心。
\item 践行宪法。将宪法原则转化为自觉的行为准则,落实在实际行动上。\\
具体来说:
\begin{enumerate}
\item 严格遵守宪法和法律规定,学会运用宪法精神来分析和解决学习和生活中的实际问题
\item 坚决维护宪法的权威,自觉抵制各种妨碍宪法实施、损害宪法尊严的行为。
\end{enumerate}
\end{enumerate}
\subsubsection{为什么要保障宪法实施?}
\label{sec:org356a2ed}
\begin{enumerate}
\item 宪法的生命在于实施,宪法的权威也在于实施。维护宪法权威需要我们保障宪法实施。
\item 如果宪法得不到充分、有效实施,受到漠视、削弱甚至破坏,国家权力的行使将会违反
法治的要求,我们的幸福生活就不能真正实现。
\item 依法治国首先是依宪治国。保障宪法实施是建设社会主义法治国家的需要。
\end{enumerate}
\subsubsection{如何保障宪法的实施?}
\label{sec:orge4918a8}
\begin{enumerate}
\item 保障宪法的实施,必须坚持依宪治国。
\item 保障宪法的实施,要加强宪法监督。
\item 保障宪法的实施,需要我们每个人增强宪法意识,尊重宪法,维护宪法。
\end{enumerate}
\section{公民权利}
\label{sec:orgd186298}
\subsection{公民基本权利}
\label{sec:orgf54e3b7}
\subsubsection{我国公民基本权利有哪些?}
\label{sec:org31dfae7}
\begin{enumerate}
\item 政治权利
\item 人身自由
\item 社会经济与文化教育权利
\item 我国公民还享有平等权、宗教信仰自由等权利,妇女儿童和残疾人等特定人群的权利受
到宪法和法律的特殊保障
\end{enumerate}
\subsubsection{公民享有政治权利的内容和意义分别是什么?}
\label{sec:org5bf4d10}
\begin{enumerate}
\item 内容:选举权和被选举权、言论、出版等自由和监督权等
\item 意义
\begin{enumerate}
\item 有利于公民依法行使政治权利,参与国家政治生活,通过各种途径和形式管理国家事
务,管理经济和文化事业,管理社会事务。
\item 是人民行使当家做主权利的重要形式
\end{enumerate}
\end{enumerate}
\subsubsection{公民行使选举权和被选举权的条件和意义分别是什么?}
\label{sec:org1f6b6f0}
\begin{enumerate}
\item 条件:
\begin{enumerate}
\item 年满18周岁的我国公民
\item 没有被依法剥夺政治权利
\end{enumerate}
\item 意义:选举权和被选举权是公民的一项基本政治权利,行使这项权利是公民参与管理国
家和管理社会的基础
\end{enumerate}
\subsubsection{言论、出版等自由的内容和重要性分别是什么?}
\label{sec:orgcbb4251}
\begin{enumerate}
\item 内容:我国公民享有言论、出版、集会、结社\footnote{让我想起《中华民国临时约法》}、游行、示威的自由。
\item 重要性:有助于公民参与国家政治生活,充分表达自己的意愿。
\end{enumerate}
\subsubsection{监督权的内容、行使原则和重要性分别是什么?}
\label{sec:org8ce4a0a}
\begin{enumerate}
\item 内容:我国公民对于任何国家机关和国家工作人员,有提出批评和建议的权利;对于任
何国家机关和国家工作人员的违法失职行为,有向有关国家机关提出申诉、控告或检举
的权利。
\item 行使原则:依法通过各种途径和形式行使监督权,不得捏造或者歪曲事实进行诬告陷害
\item 重要性:有助于国家机关及其工作人员依法行使权力,全心全意为人民服务
\end{enumerate}
\subsubsection{人身自由的含义、内容及意义分别是什么?}
\label{sec:org2118850}
\begin{enumerate}
\item 含义:指公民的人身不受非法侵犯的自由
\item 内容:包括人身自由不受侵犯,人格尊严不受侵犯,住宅不受侵犯,通信自由和通信秘
密受法律保护
\item 意义:人身自由是公民最基本、最重要的权利,只有在人身自由得到保障的前提下,公
民才能独立、自由、有尊严地生活
\end{enumerate}
\subsubsection{宪法对人身自由权作出怎样的规定?}
\label{sec:org2a868df}
\begin{enumerate}
\item 人身自由不受侵犯。我国宪法规定,任何公民,非经人民检察院批准或者人民法院决定,
并由公安机关执行,不受逮捕。禁止非法拘禁和以其他方法非法剥夺或者限制公民的人
身自由,禁止非法搜查公民的身体。
\item 人格尊严不受侵犯。我国宪法规定,公民的人格尊严不受侵犯,禁止用任何方法对公民
进行侮辱、诽谤和诬告陷害。公民的人格尊严权包括名誉权、荣誉权、肖像权、姓名权、
隐私权等。
\item 住宅不受侵犯。我国宪法规定,公民的住宅不受侵犯,禁止非法或者非法入侵公民的住
宅。通信自由和通信秘密受到法律保护。我国宪法规定,公民的通信自由和通信秘密受
到法律保护。除因国家安全或者追查刑事罪犯的需要,由公安机关或者检察机关依照法
律规定的程序对通信进行检查外,任何组织或者个人不得以任何理由侵犯公民的通信自
由和通信秘密。
\end{enumerate}
\subsubsection{社会经济权利(有哪些?)}
\label{sec:orgf92f77f}
\begin{enumerate}
\item 财产权
\begin{enumerate}
\item 重要性:我们的生存和发展及物质和文化生活需要的满足,都离不开财产
\item 宪法规定:公民的合肥的私有财产不受侵犯
\item 内容:占有权、使用权、收益权、处分权
\end{enumerate}
\item 劳动权
\begin{enumerate}
\item 含义:一切有劳动能力的公民有劳动就业和取得劳动报酬的权利
\item 重要性
\begin{enumerate}
\item 是公民赖以生存的基础
\item 可以保障人们合理的生活水平,实现自身价值,为国家和社会作出贡献
\end{enumerate}
\item 宪法规定:公民有劳动的权利和义务
\end{enumerate}
\item 物质帮助权
\begin{enumerate}
\item 宪法规定:公民在年老、疾病或者丧失劳动能力的情况下,有从国家和社会获得物质
帮助的权利
\item 保障措施:国家发展为公民享受这些权利所需要的社会保障、社会救济和医疗卫生事
业
\end{enumerate}
\end{enumerate}
\subsubsection{文化教育权利(有哪些?)}
\label{sec:org1c664fc}
\begin{enumerate}
\item 受教育权
\begin{enumerate}
\item 内涵:公民有按照其能力平等地从国家获得接受教育的机会,并获得相应物质保障的
权利
\item 重要性:为个人人生幸福奠定基础,为人类文明传递薪火,成就民族和国家的未来
\item 保障措施
\begin{enumerate}
\item 国家实行义务教育制度,保障所有失灵儿童、少年接受义务教育
\item 国家制定资助政策,不让一个学生因家庭经济困难而失学,努力让每个孩子都能
享有公平而有质量的教育
\end{enumerate}
\end{enumerate}
\item 文化权利
\begin{enumerate}
\item 宪法规定:中华人民共和国公民有进行科学研究、文学艺术创作和其他文化活动的自
由
\item 保障措施:国家对于从事教育、科学、技术、文学、艺术和其他文化事业的公民的有
益于人民的创造性工作,给以鼓励和帮助\footnote{如:表彰科学研究、组织艺术表演、修建阅览室等}
\end{enumerate}
\end{enumerate}
\subsection{依法行使权利}
\label{sec:org263864f}
\subsubsection{公民应该如何正确行使权利?}
\label{sec:org651fa0f}
\begin{enumerate}
\item 公民行使权利不能超越它本身的界限,不能滥用权利
\item 公民在行使自由和权利的时候,不得损害国家的、社会的、集体的利益和其他公民的合
法的自由和权利
\item 公民行使权利应依照法定程序,按照规定的活动方式、步骤和过程进行
\item 公民权利如果受到损害,要依照法定程序维护权利
\end{enumerate}
\subsubsection{公民行使权利遵守正当的程序的意义是什么?}
\label{sec:org653403c}
遵守正当的程序,有利于公民实际享受权利,有效避免和化解纠纷,有利于维护权益
\subsubsection{公民权利受到损害,怎样依照法定程序维护权利?(维护权利的方式有那些?)}
\label{sec:org367cee7}
\begin{enumerate}
\item 和解
\begin{enumerate}
\item 含义:当事人之间通过协商自行解决纠纷
\item 适用范围:一些常见的消费、劳动争议和交通事故纠纷等
\item 具体内容:当事人在自愿、互谅的基础上,依据法律,通过直接对话,摆事实、讲道
理,分清责任,达成协议,使纠纷得以解决
\end{enumerate}
\item 调解
\begin{enumerate}
\item 含义:通过调解组织解决纠纷
\item 适用范围:广泛用于调解纠纷
\item 具体内容:调解人以国家法律法规和政策及社会功德为依据,对纠纷双方进行疏导、
劝说,促使他们相互谅解,进行协商,资源达成协议,解决纠纷。我国调解方式主要
有 \uline{人民调解} \uline{行政调解} 和 \uline{司法调解}
\end{enumerate}
\item 仲裁
\begin{enumerate}
\item 含义:通过仲裁机构解决纠纷
\item 适用范围:公民与其他个人或组织之间发生合同纠纷和其他财产争议时,可以申请仲
裁
\item 具体内容:当事人根据他们之间订立的仲裁协议,自愿将其争议提交仲裁,并受仲裁
裁决的约束
\end{enumerate}
\item 诉讼
\begin{enumerate}
\item 含义:通过人民法院解决纠纷
\item 适用范围:如果受到非法侵害后采取其他方式不能解决问题,或者认定只有通过诉讼
途径才能维护合法权益
\item 具体内容:
\begin{enumerate}
\item 公民遇到人身关系或财产关系的争议,可以向人民法院提起民事诉讼
\item 公民对于某些侵犯自己人身、财产权的行为,可以向人民法院提起刑事诉讼
\item 公民认为行政机关的行政行为侵犯了自己的权益,可以向人民法院提起行政诉讼
\end{enumerate}
\end{enumerate}
\end{enumerate}
\section{公民义务}
\label{sec:org460fb68}
\subsection{公民基本义务}
\label{sec:org5c98240}
\subsubsection{我国公民的基本义务有那些?}
\label{sec:org7be7cd0}
\begin{enumerate}
\item 遵守宪法和法律
\item 维护国家利益
\item 依法服兵役
\item 依法纳税
\item 我国宪法阿海规定了公民应履行的其他义务,包括劳动的义务、受教育的义务、夫妻双
方实行计划剩余的义务、父母抚养教育未成年子女的义务和成年子女赡养父母的义务等
\end{enumerate}
\subsubsection{公民要遵守宪法和法律的原因、表现及做法分别是什么?}
\label{sec:orgfb2976a}
\begin{enumerate}
\item 原因:我国宪法和法律是全国各族人民意志和利益的集中体现,维护宪法和法律的尊严
是公民对国家和社会应尽的职责
\item 表现:保守国家秘密,爱护公共财产,遵守劳动纪律,遵守公共秩序,尊重社会公德
\item 做法:忠于宪法,维护宪法尊严,保障宪法实施
\begin{enumerate}
\item 自觉做到尊法学法守法用法,共同营造守法光荣,违法可耻的社会氛围
\item 自觉学习法律知识,了解法律程序规定,以法律来指导和约束自己的行为,做到依法
办事。
\end{enumerate}
\end{enumerate}
\subsubsection{维护国家统一和全国各族团结的原因和做法分别是什么?}
\label{sec:org64e1eae}
\begin{enumerate}
\item 原因:我国是统一的多民族国家,国家的统一和民族的团结,是我国顺利进行社会主义
现代化建设的基本保证
\item 做法:每个公民都应当把自己的命运与国家盛衰、民族兴亡紧密联系在一起,自觉维护
国家领土完整和主权统一,维护民族之间平等团结互助和谐的关系
\end{enumerate}
\subsubsection{维护国家安全、荣誉和利益的内容和重要性分别是什么?}
\label{sec:org16be49b}
\begin{enumerate}
\item 内容
\begin{enumerate}
\item 维护国家安全包括维护国家的主权、领土完整不受侵犯,国家秘密不被窃取、泄露和
出卖,社会秩序不被破坏等
\item 维护国家荣誉包括维护国家的尊严不受侵犯,国家的荣誉不受玷污。
\item 尾货国家利益包括维护国家的政治、经济和安全等各方面的利益
\end{enumerate}
\item 重要性:尾货国家安全、荣誉和利益是每个公民义不容辞的责任
\end{enumerate}
\subsubsection{依法服兵役}
\label{sec:org4562f4d}
\begin{enumerate}
\item 原因:
\begin{enumerate}
\item 保卫祖国、抵抗侵略是公民的神圣职责
\item 依照法律服兵役和参加民兵组织是公民的光荣义务
\end{enumerate}
\item 我国的兵役制度:我国兵役法规定,我国实行以志愿兵役为主体的志愿兵役与义务兵役
相结合的兵役制度。
\item 兵役种类:分为现役和预备役
\item 要求
\begin{enumerate}
\item 军人:必须遵守军队的条令和条例,忠于职守,随时为保卫祖国而战斗
\item 预备役人员:必须按照规定参加军事训练,随时准备应召参战,保卫祖国
\end{enumerate}
\end{enumerate}
\subsubsection{依法纳税}
\label{sec:orgb69420f}
\begin{enumerate}
\item 必要性:税收是国家财政收入的主要来源,依法纳税是公民的一项基本义务
\item 违反依法纳税义务的行为:逃税、欠税、骗税、抗税
\item 违反依法纳税义务要承担的责任:任何违反依法纳税的行为都是违法行为,情节严重、
构成犯罪的要依法追究刑事责任
\end{enumerate}
\subsection{依法履行义务}
\label{sec:org6a30a83}
\subsubsection{我国公民权利与义务的关系是怎样的?}
\label{sec:org2161330}
\begin{enumerate}
\item 公民的权利与义务相互依存、相互促进。权利的实现需要义务的履行,义务的履行促进
权利的实现
\item 公民既是合法权利的享有者,又是法定义务的承担者。任何公民享有权利的同时,必须
履行义务。
\item 公民的某些权利同时也是义务\footnote{如:劳动、受教育}
\end{enumerate}
\subsubsection{为什么要坚持权利和义务相统一?}
\label{sec:orgdecd04c}
\begin{enumerate}
\item 公民权利的充分实现,可以激发公民主人翁意识,调动其履行义务的积极性和主动性,
自觉承担对国家和社会的责任
\item 公民自觉履行义务,促进国家发展和社会进步,又为其权利的实现提供和创造了更好的
条件
\end{enumerate}
\subsubsection{如何坚持权利和义务相统一?}
\label{sec:org57ab29b}
\begin{enumerate}
\item 任何公民不能只享受权利而不承担义务,也不应只承担义务而不享受权利。
\item 要增强权利意识,依法行使权利;也要增强义务观念,自觉履行法定的义务。
\end{enumerate}
\subsubsection{为什么必须履行法定义务?}
\label{sec:orgc53b752}
\begin{enumerate}
\item 法定义务是由我国宪法和法律规定的,具有强制性
\item 自觉履行法定义务,是公民不可推卸的责任
\item 公民违反法定义务,必须依法承担相应的法律责任
\end{enumerate}
\subsubsection{我们如何履行法定义务?}
\label{sec:org47bcf2d}
\begin{enumerate}
\item 法律要求做的必须去做
\item 法律禁止做的坚决不做
\end{enumerate}
\subsubsection{违反法定义务的行为的含义及其后果是什么?}
\label{sec:orgbc54175}
\begin{enumerate}
\item 含义:公民实施了法律所禁止的行为,或者没有实施法律要求做的行为,都是违反法定
义务的行为
\item 后果:违反法定义务,必须依法承担相应的法律责任
\begin{enumerate}
\item 公民违反民事法律,应当依法承担民事责任
\item 公民违反行政法律,应当依法承担行政责任
\item 公民违法刑事法律,构成犯罪的,应当依法承担刑事责任
\end{enumerate}
\end{enumerate}
\section{我国的政治和经济制度}
\label{sec:org8555595}
\subsection{根本政治制度}
\label{sec:org3336e44}
\subsubsection{地位?}
\label{sec:org9601bfb}
人民代表大会制度是我国的根本政治制度\footnote{人大是根本制度,政治协商制度、民族区域自治制度等等都是基本制度(非根本)
\emph{(历史的DNA动了)}}
\subsubsection{人民代表大会制度的基本内容是什么?}
\label{sec:org9de61cd}
\begin{enumerate}
\item 国家的一切权力属于人民
\item 人民通过民主轩主选出代表,组成各级人民代表大会作为国家权力机关
\item 由人民代表大会产生国家行政机关、检察机关、审判机关、检察机关,这些国家权力机
关依法行使各自的职权,并对人民代表大会负责,受人民代表大会监督
\item 实行民主集中制,重大问题经人民代表大会充分讨论,遵循少数服从多数原则,民主决
定
\end{enumerate}
\subsubsection{各级人大代表的职权、职权来源和义务分别是什么?}
\label{sec:orgc534f78}
\begin{enumerate}
\item 职权:依法审议各项议案和报告 \emph{(审议权)} 、表决各项决定 \emph{(表决权)} 、提出
议案 \emph{(提案权)} 和质询案 \emph{(质询权)} 。
\item 职权来源:各级人民代表大会代表宪法和法律赋予本级人民代表大会的各项职权,参加
行使国家权力
\item 义务:人大代表必须与人民群众保持密切联系,听取和反映人民群众的意见和要求,努
力为人民服务,对人民负责,并接受人民监督。
\end{enumerate}
\subsubsection{为什么要坚持和完善人民代表大会制度?(人民代表大会制度的优越性体现在哪里?)}
\label{sec:org6f6c98a}
\begin{enumerate}
\item 实践充分证明,人民代表大会制度是符合中国国情和实际、体现社会主义国家性质、保
证人民当家做主、保障实现中华民族伟大复兴的好制度。
\item 人民代表大会制度是坚持党的领导、人民当家做主、依法治国有机统一的根本政治制度
安排。
\end{enumerate}
\subsubsection{如何坚持和完善人民代表大会制度?}
\label{sec:org9f3ef6c}
\begin{enumerate}
\item 必须毫不动摇坚持中国共产党的领导,依据人民代表大会制度,使党的主张通过法定程
序成为国家意志。
\item 必须保证和发展人民当家做主,支持和保证人民通过人民代表大会行使国家权力,扩大
人民民主,健全民主制度,丰富民主形式,拓宽民主渠道
\item 必须全面推进依法治国,通过人民代表大会制度,弘扬社会主义法治精神,实现国家各
项工作法治化
\item 必须坚持民主集中制,人民代表大会统一行使国家权力,国家机关及有合理分工又有相
互协调,保证国家统一高效组织推进各项事业
\end{enumerate}
\subsection{基本政治制度——中国共产党领导的多党合作和政治协商制度}
\label{sec:org7f30720}
\subsubsection{什么是爱国统一战线?}
\label{sec:org05907e0}
由中国共产党领导的,有各民主党派和各人民团体参加的,包括全体社会主义劳动者、社会
主义事业的建设者、拥护社会主义的爱国者、拥护祖国统一和致力于中华民族伟大复兴的爱
国者的广泛的爱国统一战线
\subsubsection{中国共产党各民主党派实行的基本方针是什么?}
\label{sec:orgb7b820b}
长期共存、互相监督、肝胆相照、荣辱与共\footnote{历史突然又活过来开始攻击我(八下历史第一课里就有)}
\subsubsection{人民政治协商会议(人民政协)的地位、主题和职能分别是什么?}
\label{sec:org7337f37}
\begin{enumerate}
\item 地位:是中国共产党领导的多党合作和政治协商的重要机构,是中国人民爱国统一战线
组织。
\item 主题:团结和民主
\item 职能:政治协商、民主监督、参政议政
\end{enumerate}
\subsubsection{坚持中国共产党领导的多党合作和政治协商制度的意义是什么?}
\label{sec:org5e705a4}
\begin{enumerate}
\item 是发扬社会主义民主的重要形式
\item 有利于反映民意,集中民智,促进科学民主决策
\item 有利与协调关系,化解矛盾,维护社会稳定和谐
\item 有利于凝聚人心,反对分裂,推进祖国和平统一大业
\end{enumerate}
\subsection{基本政治制度——民族区域自治制度}
\label{sec:org36bdb56}
\subsubsection{为什么要实行民族区域自治制度?}
\label{sec:orgb73fe2f}
\begin{enumerate}
\item 中华人民共和国是全国各族人民共同缔造的统一的多民族国家
\item 我们辽阔的疆域是各民族共同开拓的,我们悠久的历史是各民族共同书写的,我们灿烂
的文化是各民族共同创造的,我们伟大的精神是各民族共同培育的
\item 五十六个民族在长期的交往、交流、交融中确立了平等团结互助和谐的社会主义民族关
系
\end{enumerate}
\subsubsection{我国的民族区域自治制度是如何实施的?(我国的少数民族是如何实现区域自治的?)}
\label{sec:org0c57d04}
\begin{enumerate}
\item 无偶国宪法规定,各少数民族聚居的地方实行区域自治,设立自治机关,行使自治权。
我国民族自治地方分为自治区、自治州、自治县三集。 \emph{(自治地方)}
\item 民族自治地方的人民代表大会和人民政府是自治机关。 \emph{(自治机关)}
\item 自治机关在行使一般地方国家机关职权的同时,依法行使自治权,即根据本地方、本民
族政治、经济、社会、文化等方面的特点,自主管理本地方、本民族的内部事务。
\emph{(自治权)}
\item 我国民族区域自治是在国家统一领导下的自治,各民族自治地方是国家不可分割的组成
部分,民族自治机关必须服从中央的领导。 \emph{(民族自治地方和国家的关系)}
\end{enumerate}
\section{To Be Continuied}
\label{sec:org9a94ff6}
\section{其他阅读版本}
\label{sec:org88e5bb6}
\href{./八下早读材料/八下早读材料.pdf}{这里}是此文件的Pdf阅读版本

\href{./八下早读材料/八下早读材料.tex}{这里}是此文件的Latex源码版本
\end{document}